\message{ !name(bohmianMech.tex)}% fancytikzposter.tex, version 2.1
% Original template created by Elena Botoeva [botoeva@inf.unibz.it], June 2012
% 
% This file is distributed under the Creative Commons Attribution-NonCommercial 2.0
% Generic (CC BY-NC 2.0) license
% http://creativecommons.org/licenses/by-nc/2.0/ 


\documentclass{a0poster}

\usepackage{fancytikzposter} 


%%%%%%%%%%%%%%%CUSTOM HEADER
\renewcommand{\Re}{\operatorname{Re}}
\renewcommand{\Im}{\operatorname{Im}}
\usepackage{braket}
\newcommand{\mean}[1]{\langle #1 \rangle}
\usepackage{titlesec}
\titlespacing\section{0pt}{12pt plus 0pt minus 0pt}{0pt plus 0pt minus 0pt}
\usepackage{graphicx}
\bibliographystyle{plain} %apsrev4-1}
%%%%%%%%%%%%%%%%%%%%%%%%%%%%%%


%%%%% --------- Change here if you want ---------- %%%%%
%% margin for the geometry package, must be changed before using the geometry package
%% default value is 4cm
\setmargin{0.8}

%% the space between the blocks
%% default value is 2cm
\setblockspacing{0.45}

%% the height of the title stripe in block nodes, decrease it to save space
%% default value is 3cm
% \setblocktitleheight{3}

%% the number of columns in the poster, possible values 2,3
%% default value is 2
\setcolumnnumber{3}

%% the space between two or more groups of authors from different institutions
%% used in \maketitle
% \setinstituteshift{10}

%% which template to use
%% N1 simple, standard look, with a colored background and gray boxes
%% N2 board with nodes
%% N3 another standard look
%% N4 envelope-like look
%% N5 with a wave-like head, original idea taken from
%%%% http://fc09.deviantart.net/fs71/f/2010/322/1/1/scientific_poster_by_nabuy-d333ria.jpg
\usetemplate{2}

%% components of the templates
%% (the maximal possible numbers are mentioned as the parameters)
% \usecolortemplate{4}
% \usebackgroundtemplate{5}
% \usetitletemplate{2}
% \useblocknodetemplate{5}
% \useplainblocktemplate{4}
% \useinnerblocktemplate{2}


%% the height of the head drawing on top 
%% applicable to templates N3, 4 and 5
% \setheaddrawingheight{14}


%% change the basic colors
%\definecolor{myblue}{HTML}{008888} 
%\setfirstcolor{myblue}% default 116699
%\setsecondcolor{gray!80!}% default CCCCCC
%\setthirdcolor{red!80!black}% default 991111

%% change the more specific colors
% \setbackgrounddarkcolor{colorone!70!black}
% \setbackgroundlightcolor{colorone!70!}
% \settitletextcolor{textcolor}
% \settitlefillcolor{white}
% \settitledrawcolor{colortwo}
% \setblocktextcolor{textcolor}
% \setblockfillcolor{white}
% \setblocktitletextcolor{colorone}
% \setblocktitlefillcolor{colortwo} %the color of the border
% \setplainblocktextcolor{textcolor}
% \setplainblockfillcolor{colorthree!40!}
% \setplainblocktitletextcolor{textcolor}
% \setplainblocktitlefillcolor{colorthree!60!}
% \setinnerblocktextcolor{textcolor}
% \setinnerblockfillcolor{white}
% \setinnerblocktitletextcolor{white}
% \setinnerblocktitlefillcolor{colorthree}




%%% size of the document and the margins
%% A0
% \usepackage[margin=\margin cm, paperwidth=118.9cm, paperheight=84.1cm]{geometry} 
\usepackage[margin=\margin cm, paperwidth=84.1cm, paperheight=118.9cm]{geometry}
%% B1
% \usepackage[margin=\margin cm, paperwidth=70cm, paperheight=100cm]{geometry}



%% changing the fonts
\usepackage{cmbright}
%\usepackage[default]{cantarell}
%\usepackage{avant}
%\usepackage[math]{iwona}
\usepackage[math]{kurier}
\usepackage[T1]{fontenc}


%% add your packages here
\usepackage{hyperref}





\title{Bohmian Mechanics and Contextuality in (q,p)}
\author{A. S. Arora\\
  Supervisor: Prof. Arvind; QCQI Group, IISER Mohali, India %\\
  % \texttt{ms11003@iisermohali.ac.in}
}


\begin{document}

\message{ !name(bohmianMech.tex) !offset(68) }
    \section*{Formalism}
    According to Bohm's original formulation of BM, a particle is associated with (1) a position and momentum $(q,p)$, precisely and continuously defined \& (2) a wave ($\psi$). For their description, the following are assumed:
    \begin{itemize}
    \item The $\psi$-field satisfies the Schr\"odinger equation.
    \item The particle momentum is restricted to $mv=p=\nabla S=\hbar \Im(\nabla \psi/\psi) $, where $\psi=Re^{iS/\hbar}$ and $\Im$ is the imaginary part.
    \item In practice, we don't control/predict precise locations of the particle; instead we have a statistical ensemble with probability densities $\rho(q)=|\psi(q)|^2$.
    \end{itemize}

    \vspace{0.5 mm}
    Comments:

    (1) Note that the observers play no fundamental role in the formalism. If $\hbar = 0$ then we recover the classical Hamilton-Jacobi equation. Unlike QM, BM has a clear classical limit. 

    (2) These are readily generalized for $N$ particles. Non locality in that case becomes explicit; $p_i=\nabla_iS(q_1,q_2,\dots,q_N)$ viz. momentum of the $i^{th}$ particle depends on the instantaneous positions of all particles.

    (3) Extension to spins: In BM, the particle only has $(q,p)$. The spin is associated only with the wave-function. For a spinor, say $\Psi \equiv (\psi_+,\psi_-)^T$, the generalization is that $mv=\hbar \Im ( (\Psi,\nabla \Psi)/(\Psi,\Psi))$ where $(.,.)$ represents inner product in the spin space $\mathbb C ^2$.
 }

  % }

 \startsecondcolumn 

 % \blocknode
 % {}
 % {

 \blocknode
 {Determinism}
 {
\message{ !name(bohmianMech.tex) !offset(561) }

\end{document}




