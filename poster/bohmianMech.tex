% fancytikzposter.tex, version 2.1
% Original template created by Elena Botoeva [botoeva@inf.unibz.it], June 2012
% 
% This file is distributed under the Creative Commons Attribution-NonCommercial 2.0
% Generic (CC BY-NC 2.0) license
% http://creativecommons.org/licenses/by-nc/2.0/ 


\documentclass{a0poster}

\usepackage{fancytikzposter} 


%%%%% --------- Change here if you want ---------- %%%%%
%% margin for the geometry package, must be changed before using the geometry package
%% default value is 4cm
% \setmargin{4}

%% the space between the blocks
%% default value is 2cm
% \setblockspacing{2}

%% the height of the title stripe in block nodes, decrease it to save space
%% default value is 3cm
% \setblocktitleheight{3}

%% the number of columns in the poster, possible values 2,3
%% default value is 2
% \setcolumnnumber{3}

%% the space between two or more groups of authors from different institutions
%% used in \maketitle
% \setinstituteshift{10}

%% which template to use
%% N1 simple, standard look, with a colored background and gray boxes
%% N2 board with nodes
%% N3 another standard look
%% N4 envelope-like look
%% N5 with a wave-like head, original idea taken from
%%%% http://fc09.deviantart.net/fs71/f/2010/322/1/1/scientific_poster_by_nabuy-d333ria.jpg
\usetemplate{2}

%% components of the templates
%% (the maximal possible numbers are mentioned as the parameters)
% \usecolortemplate{4}
% \usebackgroundtemplate{5}
% \usetitletemplate{2}
% \useblocknodetemplate{5}
% \useplainblocktemplate{4}
% \useinnerblocktemplate{2}


%% the height of the head drawing on top 
%% applicable to templates N3, 4 and 5
% \setheaddrawingheight{14}


%% change the basic colors
%\definecolor{myblue}{HTML}{008888} 
%\setfirstcolor{myblue}% default 116699
%\setsecondcolor{gray!80!}% default CCCCCC
%\setthirdcolor{red!80!black}% default 991111

%% change the more specific colors
% \setbackgrounddarkcolor{colorone!70!black}
% \setbackgroundlightcolor{colorone!70!}
% \settitletextcolor{textcolor}
% \settitlefillcolor{white}
% \settitledrawcolor{colortwo}
% \setblocktextcolor{textcolor}
% \setblockfillcolor{white}
% \setblocktitletextcolor{colorone}
% \setblocktitlefillcolor{colortwo} %the color of the border
% \setplainblocktextcolor{textcolor}
% \setplainblockfillcolor{colorthree!40!}
% \setplainblocktitletextcolor{textcolor}
% \setplainblocktitlefillcolor{colorthree!60!}
% \setinnerblocktextcolor{textcolor}
% \setinnerblockfillcolor{white}
% \setinnerblocktitletextcolor{white}
% \setinnerblocktitlefillcolor{colorthree}




%%% size of the document and the margins
%% A0
% \usepackage[margin=\margin cm, paperwidth=118.9cm, paperheight=84.1cm]{geometry} 
\usepackage[margin=\margin cm, paperwidth=84.1cm, paperheight=118.9cm]{geometry}
%% B1
% \usepackage[margin=\margin cm, paperwidth=70cm, paperheight=100cm]{geometry}



%% changing the fonts
\usepackage{cmbright}
%\usepackage[default]{cantarell}
%\usepackage{avant}
%\usepackage[math]{iwona}
\usepackage[math]{kurier}
\usepackage[T1]{fontenc}


%% add your packages here
\usepackage{hyperref}





\title{Bohmian Mechanics and Contextuality}
\author{Atul Singh Arora\\
  QCQI Group, IISER Mohali, India\\
  \texttt{ms11003@iisermohali.ac.in}
}


\begin{document}

%%%%% ---------- the background picture ---------- %%%%%
%% to change it modify the macro \BackgroundPicture
\ClearShipoutPicture
\AddToShipoutPicture{\BackgroundPicture}

\noindent % to have the picture right in the center
\begin{tikzpicture}
  \initializesizeandshifts
  % \setxshift{15}
  % \setyshift{2}


  %% the title block, #1 - shift, the default value is (0,0), #2 - width, #3 - scale
  %% the alias of the title block is `title', so we can refer to its boundaries later
  \ifthenelse{\equal{\template}{1}}{ 
    \titleblock{47}{1}
  }{
    \titleblock{47}{1.5}
  }

  %% a logo can be added to the title block
  %% #1 - anchor relative to the title block, #2 - shift, #3 - width, #3 - file name
  % \ifthenelse{\equal{\template}{2}}{ 
  %   \addlogo[south west]{(2,0)}{6cm}{unibz_b.png}
  % }{
  %   \addlogo[south west]{(2,0)}{6cm}{unibz_w.png}
  % }


  %% a block node, with the specified position (optional), title and the content
  %% #1 - where (optional), #2 - title, #3 - text
  %%%%%%%%%% ------------------------------------------ %%%%%%%%%%
  \blocknode
  {Prerequisites}
  {I am not assuming you have any special knowledge about the subject; only undergrad level quantum mechanics is assumed}

  \blocknode%
  {Background | Bohmian Mechanics}%
  {The Quantum Mechanics that is taught, is usually the one which uses the 'Copenhagen interpretation'. This interpretation asserts that the most complete possible specification of an individual system, is in terms of $\psi$ which yields only probabilistic results. While it can be shown to be consistent, it is worth exploring the reasons for believing this assertion. David Bohm\footnote{Historically, de Broglie had formulated a similar theory and then gave it up until Bohm independently re-discovered it} in an attempt to investigate the truth constructed a theory with `hidden variables' that in principle completely specified the system but in practice get averaged over. He was able to show that his theory yields the same results as Quantum Mechanics in all the physical situations he considered.

Orthodox view: No precise description of nature is possible. \\
Bohmian Mechanics: A description of nature that is precise.}


    % \coloredbox{colorthree!50!}{
    %   \textbackslash documentclass\{a0poster\}\\
    %   \textbackslash usepackage\{fancytikzposter\} \% here most of the things are
    %   defined \\

    %   \% change parameters only after this line\\

    %   \textbackslash usepackage{\small[margin=\textbackslash margin \ cm,
    %     paperwidth=84.1cm, paperheight=118.9cm]\{ geometry\}} \\

    %   \textbackslash title\{Title\}\\
    %   \textbackslash author\{Author\textbackslash\textbackslash
    %   Institution\textbackslash\textbackslash
    %   \textbackslash texttt\{email\}\}\\
    %   \textbackslash begin\{document\}\\
    %   \textbackslash AddToShipoutPicture\{\textbackslash BackgroundPicture\}\\


    %   \textbackslash noindent \\
    %   \textbackslash begin\{tikzpicture\} \\
    %   \textbackslash initializesizeandshifts \\

    %   \textbackslash titleblock\{50\}\{1\}\\
    %   \textbackslash blocknode\{Block Title\}\{Block Content\}\\
    %   \textbackslash startsecondcolumn\\
    %   \textbackslash blocknode\{Block Title 2\}\{Block Content 2\}\\
    %   \textbackslash end\{tikzpicture\}\\
    %   \textbackslash end\{document\}
    % }
  % }


  % %% a callout block
  % %% #1 - rotate angle (optional), #2 - from, #3 - where, #4 - width, #5 - text
  % %%%%%%%%%% ------------------------------------------ %%%%%%%%%%
  % \calloutblock{($(box.center)+(-2,-8)$)}
  % {($(box.center)+(10,-1)$)}
  % {19cm}
  % {\small
  %   Macro for creating a block node:
  %   \begin{itemize}
  %   \item[] \textbackslash blocknode\{Block Title\}\{Block Content\}
  %   \end{itemize}
  %   Macro \textbackslash blocknode has three parameters. The first one is
  %   optional and it is the position of the block. The first block will be
  %   automatically placed to (\$(firstrow)-(xshift)-(yshift)\$), which is the
  %   left corner below the title block. In most of the templates, (firstrow) is
  %   set to (title.south), where \emph{title} is the alias for the title
  %   block. Each subsequent block is automatically placed to
  %   [(\$(box.south)-(yshift)\$)], i.e., below the previous block aliased
  %   \emph{box}.  You can also use an explicit parameter, e.g., $(-10,30)$ (note
  %   that (0,0) is the center of the poster). The second parameter is the title
  %   of the block. Finally, the last parameter is the  actual content. 
  % }




 %  %% by default, the position of the new block node is right below the previous
 %  %% block node, stored in (currenty)
 %  %% box is the alias of the previous block, so we can refer to its boundaries

 %  %%%%%%%%%% ------------------------------------------ %%%%%%%%%%
 %  \blocknode{Making Title}%
 %  {To make title, use the standard commands \textbackslash title and
 %    \textbackslash author in the preamble, and then the following macro:
 %    \begin{itemize}
 %    \item[] \textbackslash titleblock\{50\}\{1.5\}
 %    \end{itemize}
 %    Macro \textbackslash titleblock has three parameters. The first one is
 %    optional and it specifies the shift of the title block w.r.t.\ its default
 %    position, which is set to (\$0.5*(0,\textbackslash
 %    paperheight)-(0,\textbackslash margin)\$). The second parameter is the width
 %    of the title block, and the third parameter is the scaling ratio (to make
 %    the title bigger or smaller).\\

 %    \small
 %    The syntax for specifying authors is similar to the one in aaai.sty.  Author
 %    information can be set in various styles: For several authors from the same
 %    institution:
 %    \begin{itemize}\item[] 
 %      \textbackslash author\{Author 1 \textbackslash and ... \textbackslash and
 %      Author n \textbackslash\textbackslash\\ 
 %      Address line \textbackslash\textbackslash \
 %      ... \textbackslash\textbackslash \ Address line\}
 %    \end{itemize}

 %    If the names do not fit well on one line use
 %    \begin{itemize}\item[] 
 %      \textbackslash author\{Author 1 \textbackslash\textbackslash \
 %      \{\textbackslash bf Author 2\} \textbackslash\textbackslash \
 %      ... \textbackslash\textbackslash \
 %      \{\textbackslash bf Author n\} \textbackslash\textbackslash\\
 %      Address line \textbackslash\textbackslash \
 %      ... \textbackslash\textbackslash \ Address line\}
 %    \end{itemize}

 %    For authors from different institutions: 
 %    \begin{itemize}\item[] 
 %      \textbackslash author\{Author 1 \textbackslash\textbackslash \ Address
 %      line \textbackslash\textbackslash \ ... \textbackslash\textbackslash \
 %      Address line \\ \textbackslash And ... \textbackslash And \\
 %      Author n \textbackslash\textbackslash \ Address line
 %      \textbackslash\textbackslash \ ... \textbackslash\textbackslash \ Address
 %      line\}
 %    \end{itemize}
    
 %    To start a separate ``row'' of authors use \textbackslash AND, as in
 %    \begin{itemize}\item[] 
 %      \textbackslash author\{Author 1 \textbackslash\textbackslash \ Address line
 %      \textbackslash\textbackslash \ ... \textbackslash\textbackslash \ Address
 %      line \textbackslash AND\\ Author 2 \textbackslash\textbackslash \ Address line
 %      \textbackslash\textbackslash \ ... \textbackslash\textbackslash \ Address
 %      line \textbackslash And\\ Author 3 \textbackslash\textbackslash \ Address line
 %      \textbackslash\textbackslash \ ... \textbackslash\textbackslash \ Address
 %      line\}
 %    \end{itemize}
 %    (though, I must say \textbackslash and ... \textbackslash and did not work for
 %    me with more than 2 authors, so just use commas where you need if it does
 %    not work for you either).
 %  }
  

 %  %%%%%%%%%% ------------------------------------------ %%%%%%%%%%
 %  \blocknodew[($(currenty)-(3.5,0)$)]{30}{Variable Width Block Nodes} %
 %  { You can also create blocks of arbitrary width
 %    \begin{itemize}
 %    \item[] \textbackslash blocknodew[coordinate]\{Block width\}\{Block Title\}%
 %      \{Block Content\}
 %    \end{itemize} 
 %    % 
 %    In this case it is better to specify coordinate manually if you want to have
 %    blocks aligned vertically. \\

 %    Note that (xshift) and (yshift) are coordinates created in macro
 %    \textbackslash initializesizeandshifts, and they allow to have relative
 %    positioning of block nodes in an automatic fashion. If you want to define
 %    your own shifts, set new values for (xshift) and (yshift) using commands
 %    \textbackslash setxshift and \textbackslash setyshift.\\

 %    Also, it might be useful to know the y-coordinate of the south border of the
 %    previous block. You can retrieve it by using the command
 %    \begin{itemize}
 %    \item[] \textbackslash getcurrentrow\{box\} or \textbackslash getcurrentrow\{note\}
 %    \end{itemize}
 %    This coordinate will be stored in (currentrow), which can be used to
 %    specify the location of the next block node.
 %  }


 %  %%%%%%%%%% ------------------------------------------ %%%%%%%%%%
 %  \plainblock[5]{($(currenty)+(4,2)$)}{35}{fancyTikZposter template} %
 %  {
    
 %    \vspace{0.3cm}
 %    It is a template for scientific posters based on a0poster and TikZ
 %    only. The current version contains five (plus one) different templates (see my
 %    posters
 %    % 
 %    \href{http://www.inf.unibz.it/~ebotoeva/presentations/abcrs-KR-12-poster.pdf}{%
 %      \underline{here}} and
 %    % 
 %    \href{http://www.inf.unibz.it/~ebotoeva/presentations/boto-RR-12-poster.pdf}{%
 %      \underline{here}}). The sources of this pdf file can be found
 %    \href{http://www.inf.unibz.it/~ebotoeva/tikz/tikzposter_sources.zip}{\underline{here}}.}

  


 %  %%%%%%%%%%%%% NEW COLUMN %%%%%%%%%%%%%%% 
 %  \startsecondcolumn 

 %  %%%%%%%%%% ------------------------------------------ %%%%%%%%%%
 %  \blocknode%
 %  {Block Nodes in the Second Column}%
 %  {To start the second column or the third column use commands
 %    \begin{itemize}
 %    \item[] \textbackslash startsecondcolumn, and \textbackslash startthirdcolumn.
 %    \end{itemize}
 %    If the number of columns is 2, then the last command will not have
 %    effect. \\

 %    You can also start a new column with an arbitrary x-coordinate by specifying
 %    explicitly the coordinate of the new block node as follows:
 %    \begin{itemize}
 %    \item[] \textbackslash blocknode[(\$(firstrow)-(yshift)+(x,0)\$)]\{Block
 %      Title\}\{Block Content\}
 %    \end{itemize}

 %    % 
 %  }


  
 %  %%%%%%%%%% ------------------------------------------ %%%%%%%%%%
 %  \blocknode{Useful Macro Within Block Nodes}%
 %  {There are three types of colored boxes/blocks that you can use inside block
 %    nodes to highlight information. \\
    
 %    \begin{tabular}[t]{ll}
 %      \begin{minipage}{0.5\linewidth}
 %        \innerblock{Theorem} {Statement}
 %      \end{minipage}
 %      & 
 %      \textbackslash innerblock\{Theorem\}\{Statement\}\\

 %      \begin{minipage}{0.5\linewidth}
 %        \innerblockplain[colorone!80!]{Text}
 %      \end{minipage}
 %      &
 %      \textbackslash innerblockplain[colorone!80!]\{Text\}\\ 

 %      \begin{minipage}{0.5\linewidth}
 %        \coloredbox{colorthree!50!}{Text}
 %      \end{minipage}
 %      &
 %      \textbackslash coloredbox\{colorthree!50!\}\{Text\}
 %    \end{tabular}

 %    \vspace{0.5cm}
 %    The default figure environment does not work within a tikzpicture. I created
 %    a new figure environment that can be used instead, based on the code sent by
 %    Stephan Thober.
 %    \begin{itemize}
 %    \item[] \textbackslash begin\{tikzfigure\}[Caption]\\
 %      \ldots\\
 %      \textbackslash end\{tikzfigure\}
 %    \end{itemize} 
 %    % 

 %    \begin{tikzfigure}[A shaded circle]
 %      \begin{tikzpicture}
 %        \draw[draw=none,inner color=colorthree, outer color=colorone] (0,0) circle (2cm);
 %      \end{tikzpicture}
 %    \end{tikzfigure}
 %  }


 %  %%%%%%%%%% ------------------------------------------ %%%%%%%%%%
 %  \calloutblock{($(box.south east)-(8,-2)$)}
 %  {($(box.south east)-(16,2)$)}
 %  {30cm}
 %  {
 %    There are also callout blocks that allow for a more interesting layout of the
 %    poster. 
 %    \begin{itemize}
 %    \item[] \textbackslash calloutblock[rotate angle]\{from
 %      coordinate\}\{coordinate\}\{Block Width\}\{Block Content\} 
 %    \end{itemize}
 %    The alias for such blocks is \emph{note}.
 %  }


 %  %% to place the next node centered vertically in the second column, we can
 %  %% obtain the y-coordinate of the previous node using macro
 %  %% \getcurrentrow{note}, where note is the alias of the callout node, and
 %  %% then specify the coordinate of the next node using coordinate (currentrow)
 %  \getcurrentrow{note}


 %  %% a plain block
 %  %% #1 - rotate angle (optional), #2 - where, #3 - width, #4 - title, #5 - text
 %  %%%%%%%%%% ------------------------------------------ %%%%%%%%%%
 %  \plainblock{($(currentrow)+(xshift)-(yshift)$)}%[($(currenty)+(0,10)$)]%
 %  {32}{Plain blocks} %
 %  {These blocks are similar to callout blocks. They allow for specifying the
 %    title of the block.
 %    \begin{itemize}
 %    \item[] \textbackslash plainblock[rotate angle]\{coordinate\}\{Block Width\}\{Block
 %      Title\}\{Block Content\} 
 %    \end{itemize}
 %  }


 
 %  %% the coordinate (currenty) is used in the default placing of the next blocknode
 % \getcurrentrow{note}
 % \coordinate (currenty) at ($(currentrow)+(xshift)-(yshift)$);



 %   %%%%%%%%%%%%% NEW COLUMN %%%%%%%%%%%%%%% 
 %  %% (if column number is 3)
 %  \startthirdcolumn

 %  %%%%%%%%%% ------------------------------------------ %%%%%%%%%%
 %  \blocknode {Personalizing the Poster}%
 %  {It is possible to adjust the layout of the poster. To impose your own
 %    setting, you can use these macros:
 %    \begin{itemize}
 %    \item Macros for changing sizes
 %      \begin{itemize}
 %      \item[] \textbackslash setmargin\{4\},
 %        %% the height of the head drawing on top
 %        %% applicable to templates N2 and 4
 %        \textbackslash setheaddrawingheight\{14\},
 %        %% the space between two or more groups of authors from different
 %        %% institutions
 %        %% used in \maketitle
 %        \textbackslash setinstituteshift\{10\},\\
 %        %% the space between the blocks
 %        %% default value is 2cm
 %        \textbackslash setblockspacing\{2\},
 %        %% the height of the title stripe in block nodes, decrease it to save space
 %        %% default value is 3cm
 %        \textbackslash setblocktitleheight\{3\}
 %      \end{itemize}

 %    \item Other structural macros
 %      \begin{itemize}
 %      \item[]  %% the number of columns in the poster, possible values 2,3
 %        %% default value is 2
 %        \textbackslash setcolumnnumber\{3\},
 %        %% which template to use 
 %        %% N1 simple, standard look, with a colored background and gray boxes
 %        %% N2 board with nodes
 %        %% N3 another standard look
 %        %% N4 envelope like look
 %        %% N5 with a wave-like head, original idea taken from
 %        %%%% http://fc09.deviantart.net/fs71/f/2010/322/1/1/scientific_poster_by_nabuy-d333ria.jpg
 %        %% N6 experimental, oriental style, largely based on template N3
 %        \textbackslash usetemplate\{6\},\\
 %        \textbackslash usecolortemplate\{4\},
 %        \textbackslash usebackgroundtemplate\{5\},
 %        \textbackslash usetitletemplate\{2\},\\
 %        \textbackslash useblocknodetemplate\{5\},
 %        \textbackslash useinnerblocktemplate\{3\},
 %        \textbackslash useplainblocktemplate\{4\}

 %      \end{itemize}

 %    \item Macro for adding logos to the title block
 %      \begin{itemize}
 %      \item[] \textbackslash addlogo[south west]\{(0,0)\}\{6cm\}\{filename\}
 %      \end{itemize}

 %    \item Macros for the basic colors
 %      \begin{itemize}
 %      \item[] \textbackslash setfirstcolor\{green!70!\}, % default 116699
 %        \textbackslash setsecondcolor\{gray!80!\}, % default CCCCCC
 %        \textbackslash setthirdcolor\{red!80!black\}% default 991111
 %      \end{itemize}

 %    \item Macros for specific colors:
 %      \begin{itemize}
 %      \item[] \textbackslash setbackgrounddarkcolor\{colorone!70!black\},
 %        \textbackslash setbackgroundlightcolor\{{\small colorone!70!}\},\\
 %        \textbackslash settitletextcolor\{textcolor\},
 %        \textbackslash settitlefillcolor\{white\},
 %        \textbackslash settitledrawcolor\{colortwo\},\\
 %        \textbackslash setblocktextcolor\{textcolor\},
 %        \textbackslash setblockfillcolor\{white\},\\
 %        \textbackslash setblocktitletextcolor\{colorone\},
 %        \textbackslash setblocktitlefillcolor\{colortwo\}, \\
 %        \textbackslash setplainblocktextcolor\{textcolor\},
 %        \textbackslash setplainblockfillcolor\{colorthree!40\},\\
 %        \textbackslash setplainblocktitletextcolor\{textcolor\},
 %        \textbackslash setplainblocktitlefillcolor\{colorthree!60\}, \\
 %        \textbackslash setinnerblocktextcolor\{textcolor\},
 %        \textbackslash setinnerblockfillcolor\{white\},\\
 %        \textbackslash setinnerblocktitletextcolor\{white\},
 %        \textbackslash setinnerblocktitlefillcolor\{colorthree\},
 %      \end{itemize}
 %    \end{itemize}
  % }



\end{tikzpicture}


\end{document}




