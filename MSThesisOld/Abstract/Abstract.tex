\begin{abstract}
The Copenhagen Interpretation of Quantum Mechanics (QM) asserts that
the wavefunction is the most complete description, which entails that
there is an inherent fuzziness in our description of nature. There
exists a completion of QM, known as Bohmian Mechanics (BM), which
replaces this fuzziness with precision, and re-introduces notions
of physical trajectories. Various interesting questions arise, solely
by existence of such a description; doesn't it contradict the uncertainty
principle, for instance. Most of these questions were found to have
been addressed satisfactorily in the literature. There was however,
one question, whose answer has become the subject of the thesis; that
of the paradoxical co-existence of contextuality and BM. In a theory
that can predict the value of operators, the value an operator takes,
must depend on the state of the system (including hidden variables).
Contextuality arguments show that the value an operator takes, must
also depend on the complete set of compatible operators, to be consistent
with QM. BM being deterministic, is at complete odds with this notion. 

After various attempts we were able to show, that the notion of contextuality
is infact not necessary. This was achieved by identifying another
`classical property' and constructing a non-contextual toy-model,
serving as a counter-example to the impossibility proof. The toy model
has been generalized to a discrete but arbitrarily sized Hilbert space,
consistent with all predictions of QM. Implications of violation of
this `classical property' have been discussed, in particular, to the
notion of non-locality.

\begin{comment}
given the physical situation (including hidden variables) and the
state, can predict the value of any operator; the prediction doesn't
depend on anything else.
\end{comment}
\end{abstract}

